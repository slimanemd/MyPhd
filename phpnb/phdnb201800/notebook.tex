
% Default to the notebook output style

    


% Inherit from the specified cell style.




    
\documentclass[11pt]{article}

    
    
    \usepackage[T1]{fontenc}
    % Nicer default font (+ math font) than Computer Modern for most use cases
    \usepackage{mathpazo}

    % Basic figure setup, for now with no caption control since it's done
    % automatically by Pandoc (which extracts ![](path) syntax from Markdown).
    \usepackage{graphicx}
    % We will generate all images so they have a width \maxwidth. This means
    % that they will get their normal width if they fit onto the page, but
    % are scaled down if they would overflow the margins.
    \makeatletter
    \def\maxwidth{\ifdim\Gin@nat@width>\linewidth\linewidth
    \else\Gin@nat@width\fi}
    \makeatother
    \let\Oldincludegraphics\includegraphics
    % Set max figure width to be 80% of text width, for now hardcoded.
    \renewcommand{\includegraphics}[1]{\Oldincludegraphics[width=.8\maxwidth]{#1}}
    % Ensure that by default, figures have no caption (until we provide a
    % proper Figure object with a Caption API and a way to capture that
    % in the conversion process - todo).
    \usepackage{caption}
    \DeclareCaptionLabelFormat{nolabel}{}
    \captionsetup{labelformat=nolabel}

    \usepackage{adjustbox} % Used to constrain images to a maximum size 
    \usepackage{xcolor} % Allow colors to be defined
    \usepackage{enumerate} % Needed for markdown enumerations to work
    \usepackage{geometry} % Used to adjust the document margins
    \usepackage{amsmath} % Equations
    \usepackage{amssymb} % Equations
    \usepackage{textcomp} % defines textquotesingle
    % Hack from http://tex.stackexchange.com/a/47451/13684:
    \AtBeginDocument{%
        \def\PYZsq{\textquotesingle}% Upright quotes in Pygmentized code
    }
    \usepackage{upquote} % Upright quotes for verbatim code
    \usepackage{eurosym} % defines \euro
    \usepackage[mathletters]{ucs} % Extended unicode (utf-8) support
    \usepackage[utf8x]{inputenc} % Allow utf-8 characters in the tex document
    \usepackage{fancyvrb} % verbatim replacement that allows latex
    \usepackage{grffile} % extends the file name processing of package graphics 
                         % to support a larger range 
    % The hyperref package gives us a pdf with properly built
    % internal navigation ('pdf bookmarks' for the table of contents,
    % internal cross-reference links, web links for URLs, etc.)
    \usepackage{hyperref}
    \usepackage{longtable} % longtable support required by pandoc >1.10
    \usepackage{booktabs}  % table support for pandoc > 1.12.2
    \usepackage[inline]{enumitem} % IRkernel/repr support (it uses the enumerate* environment)
    \usepackage[normalem]{ulem} % ulem is needed to support strikethroughs (\sout)
                                % normalem makes italics be italics, not underlines
    

    
    
    % Colors for the hyperref package
    \definecolor{urlcolor}{rgb}{0,.145,.698}
    \definecolor{linkcolor}{rgb}{.71,0.21,0.01}
    \definecolor{citecolor}{rgb}{.12,.54,.11}

    % ANSI colors
    \definecolor{ansi-black}{HTML}{3E424D}
    \definecolor{ansi-black-intense}{HTML}{282C36}
    \definecolor{ansi-red}{HTML}{E75C58}
    \definecolor{ansi-red-intense}{HTML}{B22B31}
    \definecolor{ansi-green}{HTML}{00A250}
    \definecolor{ansi-green-intense}{HTML}{007427}
    \definecolor{ansi-yellow}{HTML}{DDB62B}
    \definecolor{ansi-yellow-intense}{HTML}{B27D12}
    \definecolor{ansi-blue}{HTML}{208FFB}
    \definecolor{ansi-blue-intense}{HTML}{0065CA}
    \definecolor{ansi-magenta}{HTML}{D160C4}
    \definecolor{ansi-magenta-intense}{HTML}{A03196}
    \definecolor{ansi-cyan}{HTML}{60C6C8}
    \definecolor{ansi-cyan-intense}{HTML}{258F8F}
    \definecolor{ansi-white}{HTML}{C5C1B4}
    \definecolor{ansi-white-intense}{HTML}{A1A6B2}

    % commands and environments needed by pandoc snippets
    % extracted from the output of `pandoc -s`
    \providecommand{\tightlist}{%
      \setlength{\itemsep}{0pt}\setlength{\parskip}{0pt}}
    \DefineVerbatimEnvironment{Highlighting}{Verbatim}{commandchars=\\\{\}}
    % Add ',fontsize=\small' for more characters per line
    \newenvironment{Shaded}{}{}
    \newcommand{\KeywordTok}[1]{\textcolor[rgb]{0.00,0.44,0.13}{\textbf{{#1}}}}
    \newcommand{\DataTypeTok}[1]{\textcolor[rgb]{0.56,0.13,0.00}{{#1}}}
    \newcommand{\DecValTok}[1]{\textcolor[rgb]{0.25,0.63,0.44}{{#1}}}
    \newcommand{\BaseNTok}[1]{\textcolor[rgb]{0.25,0.63,0.44}{{#1}}}
    \newcommand{\FloatTok}[1]{\textcolor[rgb]{0.25,0.63,0.44}{{#1}}}
    \newcommand{\CharTok}[1]{\textcolor[rgb]{0.25,0.44,0.63}{{#1}}}
    \newcommand{\StringTok}[1]{\textcolor[rgb]{0.25,0.44,0.63}{{#1}}}
    \newcommand{\CommentTok}[1]{\textcolor[rgb]{0.38,0.63,0.69}{\textit{{#1}}}}
    \newcommand{\OtherTok}[1]{\textcolor[rgb]{0.00,0.44,0.13}{{#1}}}
    \newcommand{\AlertTok}[1]{\textcolor[rgb]{1.00,0.00,0.00}{\textbf{{#1}}}}
    \newcommand{\FunctionTok}[1]{\textcolor[rgb]{0.02,0.16,0.49}{{#1}}}
    \newcommand{\RegionMarkerTok}[1]{{#1}}
    \newcommand{\ErrorTok}[1]{\textcolor[rgb]{1.00,0.00,0.00}{\textbf{{#1}}}}
    \newcommand{\NormalTok}[1]{{#1}}
    
    % Additional commands for more recent versions of Pandoc
    \newcommand{\ConstantTok}[1]{\textcolor[rgb]{0.53,0.00,0.00}{{#1}}}
    \newcommand{\SpecialCharTok}[1]{\textcolor[rgb]{0.25,0.44,0.63}{{#1}}}
    \newcommand{\VerbatimStringTok}[1]{\textcolor[rgb]{0.25,0.44,0.63}{{#1}}}
    \newcommand{\SpecialStringTok}[1]{\textcolor[rgb]{0.73,0.40,0.53}{{#1}}}
    \newcommand{\ImportTok}[1]{{#1}}
    \newcommand{\DocumentationTok}[1]{\textcolor[rgb]{0.73,0.13,0.13}{\textit{{#1}}}}
    \newcommand{\AnnotationTok}[1]{\textcolor[rgb]{0.38,0.63,0.69}{\textbf{\textit{{#1}}}}}
    \newcommand{\CommentVarTok}[1]{\textcolor[rgb]{0.38,0.63,0.69}{\textbf{\textit{{#1}}}}}
    \newcommand{\VariableTok}[1]{\textcolor[rgb]{0.10,0.09,0.49}{{#1}}}
    \newcommand{\ControlFlowTok}[1]{\textcolor[rgb]{0.00,0.44,0.13}{\textbf{{#1}}}}
    \newcommand{\OperatorTok}[1]{\textcolor[rgb]{0.40,0.40,0.40}{{#1}}}
    \newcommand{\BuiltInTok}[1]{{#1}}
    \newcommand{\ExtensionTok}[1]{{#1}}
    \newcommand{\PreprocessorTok}[1]{\textcolor[rgb]{0.74,0.48,0.00}{{#1}}}
    \newcommand{\AttributeTok}[1]{\textcolor[rgb]{0.49,0.56,0.16}{{#1}}}
    \newcommand{\InformationTok}[1]{\textcolor[rgb]{0.38,0.63,0.69}{\textbf{\textit{{#1}}}}}
    \newcommand{\WarningTok}[1]{\textcolor[rgb]{0.38,0.63,0.69}{\textbf{\textit{{#1}}}}}
    
    
    % Define a nice break command that doesn't care if a line doesn't already
    % exist.
    \def\br{\hspace*{\fill} \\* }
    % Math Jax compatability definitions
    \def\gt{>}
    \def\lt{<}
    % Document parameters
    \title{phd2018}
    
    
    

    % Pygments definitions
    
\makeatletter
\def\PY@reset{\let\PY@it=\relax \let\PY@bf=\relax%
    \let\PY@ul=\relax \let\PY@tc=\relax%
    \let\PY@bc=\relax \let\PY@ff=\relax}
\def\PY@tok#1{\csname PY@tok@#1\endcsname}
\def\PY@toks#1+{\ifx\relax#1\empty\else%
    \PY@tok{#1}\expandafter\PY@toks\fi}
\def\PY@do#1{\PY@bc{\PY@tc{\PY@ul{%
    \PY@it{\PY@bf{\PY@ff{#1}}}}}}}
\def\PY#1#2{\PY@reset\PY@toks#1+\relax+\PY@do{#2}}

\expandafter\def\csname PY@tok@w\endcsname{\def\PY@tc##1{\textcolor[rgb]{0.73,0.73,0.73}{##1}}}
\expandafter\def\csname PY@tok@c\endcsname{\let\PY@it=\textit\def\PY@tc##1{\textcolor[rgb]{0.25,0.50,0.50}{##1}}}
\expandafter\def\csname PY@tok@cp\endcsname{\def\PY@tc##1{\textcolor[rgb]{0.74,0.48,0.00}{##1}}}
\expandafter\def\csname PY@tok@k\endcsname{\let\PY@bf=\textbf\def\PY@tc##1{\textcolor[rgb]{0.00,0.50,0.00}{##1}}}
\expandafter\def\csname PY@tok@kp\endcsname{\def\PY@tc##1{\textcolor[rgb]{0.00,0.50,0.00}{##1}}}
\expandafter\def\csname PY@tok@kt\endcsname{\def\PY@tc##1{\textcolor[rgb]{0.69,0.00,0.25}{##1}}}
\expandafter\def\csname PY@tok@o\endcsname{\def\PY@tc##1{\textcolor[rgb]{0.40,0.40,0.40}{##1}}}
\expandafter\def\csname PY@tok@ow\endcsname{\let\PY@bf=\textbf\def\PY@tc##1{\textcolor[rgb]{0.67,0.13,1.00}{##1}}}
\expandafter\def\csname PY@tok@nb\endcsname{\def\PY@tc##1{\textcolor[rgb]{0.00,0.50,0.00}{##1}}}
\expandafter\def\csname PY@tok@nf\endcsname{\def\PY@tc##1{\textcolor[rgb]{0.00,0.00,1.00}{##1}}}
\expandafter\def\csname PY@tok@nc\endcsname{\let\PY@bf=\textbf\def\PY@tc##1{\textcolor[rgb]{0.00,0.00,1.00}{##1}}}
\expandafter\def\csname PY@tok@nn\endcsname{\let\PY@bf=\textbf\def\PY@tc##1{\textcolor[rgb]{0.00,0.00,1.00}{##1}}}
\expandafter\def\csname PY@tok@ne\endcsname{\let\PY@bf=\textbf\def\PY@tc##1{\textcolor[rgb]{0.82,0.25,0.23}{##1}}}
\expandafter\def\csname PY@tok@nv\endcsname{\def\PY@tc##1{\textcolor[rgb]{0.10,0.09,0.49}{##1}}}
\expandafter\def\csname PY@tok@no\endcsname{\def\PY@tc##1{\textcolor[rgb]{0.53,0.00,0.00}{##1}}}
\expandafter\def\csname PY@tok@nl\endcsname{\def\PY@tc##1{\textcolor[rgb]{0.63,0.63,0.00}{##1}}}
\expandafter\def\csname PY@tok@ni\endcsname{\let\PY@bf=\textbf\def\PY@tc##1{\textcolor[rgb]{0.60,0.60,0.60}{##1}}}
\expandafter\def\csname PY@tok@na\endcsname{\def\PY@tc##1{\textcolor[rgb]{0.49,0.56,0.16}{##1}}}
\expandafter\def\csname PY@tok@nt\endcsname{\let\PY@bf=\textbf\def\PY@tc##1{\textcolor[rgb]{0.00,0.50,0.00}{##1}}}
\expandafter\def\csname PY@tok@nd\endcsname{\def\PY@tc##1{\textcolor[rgb]{0.67,0.13,1.00}{##1}}}
\expandafter\def\csname PY@tok@s\endcsname{\def\PY@tc##1{\textcolor[rgb]{0.73,0.13,0.13}{##1}}}
\expandafter\def\csname PY@tok@sd\endcsname{\let\PY@it=\textit\def\PY@tc##1{\textcolor[rgb]{0.73,0.13,0.13}{##1}}}
\expandafter\def\csname PY@tok@si\endcsname{\let\PY@bf=\textbf\def\PY@tc##1{\textcolor[rgb]{0.73,0.40,0.53}{##1}}}
\expandafter\def\csname PY@tok@se\endcsname{\let\PY@bf=\textbf\def\PY@tc##1{\textcolor[rgb]{0.73,0.40,0.13}{##1}}}
\expandafter\def\csname PY@tok@sr\endcsname{\def\PY@tc##1{\textcolor[rgb]{0.73,0.40,0.53}{##1}}}
\expandafter\def\csname PY@tok@ss\endcsname{\def\PY@tc##1{\textcolor[rgb]{0.10,0.09,0.49}{##1}}}
\expandafter\def\csname PY@tok@sx\endcsname{\def\PY@tc##1{\textcolor[rgb]{0.00,0.50,0.00}{##1}}}
\expandafter\def\csname PY@tok@m\endcsname{\def\PY@tc##1{\textcolor[rgb]{0.40,0.40,0.40}{##1}}}
\expandafter\def\csname PY@tok@gh\endcsname{\let\PY@bf=\textbf\def\PY@tc##1{\textcolor[rgb]{0.00,0.00,0.50}{##1}}}
\expandafter\def\csname PY@tok@gu\endcsname{\let\PY@bf=\textbf\def\PY@tc##1{\textcolor[rgb]{0.50,0.00,0.50}{##1}}}
\expandafter\def\csname PY@tok@gd\endcsname{\def\PY@tc##1{\textcolor[rgb]{0.63,0.00,0.00}{##1}}}
\expandafter\def\csname PY@tok@gi\endcsname{\def\PY@tc##1{\textcolor[rgb]{0.00,0.63,0.00}{##1}}}
\expandafter\def\csname PY@tok@gr\endcsname{\def\PY@tc##1{\textcolor[rgb]{1.00,0.00,0.00}{##1}}}
\expandafter\def\csname PY@tok@ge\endcsname{\let\PY@it=\textit}
\expandafter\def\csname PY@tok@gs\endcsname{\let\PY@bf=\textbf}
\expandafter\def\csname PY@tok@gp\endcsname{\let\PY@bf=\textbf\def\PY@tc##1{\textcolor[rgb]{0.00,0.00,0.50}{##1}}}
\expandafter\def\csname PY@tok@go\endcsname{\def\PY@tc##1{\textcolor[rgb]{0.53,0.53,0.53}{##1}}}
\expandafter\def\csname PY@tok@gt\endcsname{\def\PY@tc##1{\textcolor[rgb]{0.00,0.27,0.87}{##1}}}
\expandafter\def\csname PY@tok@err\endcsname{\def\PY@bc##1{\setlength{\fboxsep}{0pt}\fcolorbox[rgb]{1.00,0.00,0.00}{1,1,1}{\strut ##1}}}
\expandafter\def\csname PY@tok@kc\endcsname{\let\PY@bf=\textbf\def\PY@tc##1{\textcolor[rgb]{0.00,0.50,0.00}{##1}}}
\expandafter\def\csname PY@tok@kd\endcsname{\let\PY@bf=\textbf\def\PY@tc##1{\textcolor[rgb]{0.00,0.50,0.00}{##1}}}
\expandafter\def\csname PY@tok@kn\endcsname{\let\PY@bf=\textbf\def\PY@tc##1{\textcolor[rgb]{0.00,0.50,0.00}{##1}}}
\expandafter\def\csname PY@tok@kr\endcsname{\let\PY@bf=\textbf\def\PY@tc##1{\textcolor[rgb]{0.00,0.50,0.00}{##1}}}
\expandafter\def\csname PY@tok@bp\endcsname{\def\PY@tc##1{\textcolor[rgb]{0.00,0.50,0.00}{##1}}}
\expandafter\def\csname PY@tok@fm\endcsname{\def\PY@tc##1{\textcolor[rgb]{0.00,0.00,1.00}{##1}}}
\expandafter\def\csname PY@tok@vc\endcsname{\def\PY@tc##1{\textcolor[rgb]{0.10,0.09,0.49}{##1}}}
\expandafter\def\csname PY@tok@vg\endcsname{\def\PY@tc##1{\textcolor[rgb]{0.10,0.09,0.49}{##1}}}
\expandafter\def\csname PY@tok@vi\endcsname{\def\PY@tc##1{\textcolor[rgb]{0.10,0.09,0.49}{##1}}}
\expandafter\def\csname PY@tok@vm\endcsname{\def\PY@tc##1{\textcolor[rgb]{0.10,0.09,0.49}{##1}}}
\expandafter\def\csname PY@tok@sa\endcsname{\def\PY@tc##1{\textcolor[rgb]{0.73,0.13,0.13}{##1}}}
\expandafter\def\csname PY@tok@sb\endcsname{\def\PY@tc##1{\textcolor[rgb]{0.73,0.13,0.13}{##1}}}
\expandafter\def\csname PY@tok@sc\endcsname{\def\PY@tc##1{\textcolor[rgb]{0.73,0.13,0.13}{##1}}}
\expandafter\def\csname PY@tok@dl\endcsname{\def\PY@tc##1{\textcolor[rgb]{0.73,0.13,0.13}{##1}}}
\expandafter\def\csname PY@tok@s2\endcsname{\def\PY@tc##1{\textcolor[rgb]{0.73,0.13,0.13}{##1}}}
\expandafter\def\csname PY@tok@sh\endcsname{\def\PY@tc##1{\textcolor[rgb]{0.73,0.13,0.13}{##1}}}
\expandafter\def\csname PY@tok@s1\endcsname{\def\PY@tc##1{\textcolor[rgb]{0.73,0.13,0.13}{##1}}}
\expandafter\def\csname PY@tok@mb\endcsname{\def\PY@tc##1{\textcolor[rgb]{0.40,0.40,0.40}{##1}}}
\expandafter\def\csname PY@tok@mf\endcsname{\def\PY@tc##1{\textcolor[rgb]{0.40,0.40,0.40}{##1}}}
\expandafter\def\csname PY@tok@mh\endcsname{\def\PY@tc##1{\textcolor[rgb]{0.40,0.40,0.40}{##1}}}
\expandafter\def\csname PY@tok@mi\endcsname{\def\PY@tc##1{\textcolor[rgb]{0.40,0.40,0.40}{##1}}}
\expandafter\def\csname PY@tok@il\endcsname{\def\PY@tc##1{\textcolor[rgb]{0.40,0.40,0.40}{##1}}}
\expandafter\def\csname PY@tok@mo\endcsname{\def\PY@tc##1{\textcolor[rgb]{0.40,0.40,0.40}{##1}}}
\expandafter\def\csname PY@tok@ch\endcsname{\let\PY@it=\textit\def\PY@tc##1{\textcolor[rgb]{0.25,0.50,0.50}{##1}}}
\expandafter\def\csname PY@tok@cm\endcsname{\let\PY@it=\textit\def\PY@tc##1{\textcolor[rgb]{0.25,0.50,0.50}{##1}}}
\expandafter\def\csname PY@tok@cpf\endcsname{\let\PY@it=\textit\def\PY@tc##1{\textcolor[rgb]{0.25,0.50,0.50}{##1}}}
\expandafter\def\csname PY@tok@c1\endcsname{\let\PY@it=\textit\def\PY@tc##1{\textcolor[rgb]{0.25,0.50,0.50}{##1}}}
\expandafter\def\csname PY@tok@cs\endcsname{\let\PY@it=\textit\def\PY@tc##1{\textcolor[rgb]{0.25,0.50,0.50}{##1}}}

\def\PYZbs{\char`\\}
\def\PYZus{\char`\_}
\def\PYZob{\char`\{}
\def\PYZcb{\char`\}}
\def\PYZca{\char`\^}
\def\PYZam{\char`\&}
\def\PYZlt{\char`\<}
\def\PYZgt{\char`\>}
\def\PYZsh{\char`\#}
\def\PYZpc{\char`\%}
\def\PYZdl{\char`\$}
\def\PYZhy{\char`\-}
\def\PYZsq{\char`\'}
\def\PYZdq{\char`\"}
\def\PYZti{\char`\~}
% for compatibility with earlier versions
\def\PYZat{@}
\def\PYZlb{[}
\def\PYZrb{]}
\makeatother


    % Exact colors from NB
    \definecolor{incolor}{rgb}{0.0, 0.0, 0.5}
    \definecolor{outcolor}{rgb}{0.545, 0.0, 0.0}



    
    % Prevent overflowing lines due to hard-to-break entities
    \sloppy 
    % Setup hyperref package
    \hypersetup{
      breaklinks=true,  % so long urls are correctly broken across lines
      colorlinks=true,
      urlcolor=urlcolor,
      linkcolor=linkcolor,
      citecolor=citecolor,
      }
    % Slightly bigger margins than the latex defaults
    
    \geometry{verbose,tmargin=1in,bmargin=1in,lmargin=1in,rmargin=1in}
    
    

    \begin{document}
    
    
    \maketitle
    
    

    
     Section \ref{bs}

\begin{verbatim}
<h3>PHD Thesis - Computer Science</h3>
<br>
<h1><b><span style="color:blue">
\end{verbatim}

Scheduling of Task Graph (DAG)with Data Placement on Multicore NUMA
Platform

\begin{verbatim}
<!--h4>Submitted by : </h4> <h2>Mohammed SLIMANE </h2>
<h4>Supervised by :</h4> <h2>Pr. Larbi SEKHRI</h2-->   
\end{verbatim}

\begin{longtable}[]{@{}ll@{}}
\toprule
\bottomrule
\end{longtable}

\textbar{}

Presented by :

Mohammed SLIMANE

\textbar{}

Supervised by :

Pr. Larbi SEKHRI

\textbar{} 

    Section \ref{home} - Section \ref{toc}

 

     {Section \ref{bs} - Section \ref{pb0} ~~~~ Goto : {[}
Section \ref{pb0}- Section \ref{archpar}- Section \ref{numa}-
Section \ref{dag0}- Section \ref{task}- Section \ref{dpd}-
Section \ref{sch}- Section \ref{penalitynuma}- Section \ref{state}-
Section \ref{loadbal}- Section \ref{contributions}-
Section \ref{indpdtk}- Section \ref{exehorizon}-
Section \ref{dagpartitionning}- Section \ref{xxh}- Section \ref{xxh2}-
Section \ref{xxh3}- Section \ref{xxh4}- Section \ref{dbws}-
Section \ref{sim}- Section \ref{simcombin}- Section \ref{simxxh}-
Section \ref{simxxhcmax}- Section \ref{simxxhpnuma}-
Section \ref{simdbws}- Section \ref{conclusion}-
Section \ref{productions} {]} }

Table of Content

~~~~~~Section \ref{c1} {[}1{]}. Domain \& Problematic
~~~~~~Section \ref{c2} {[}2{]}. State of Art ~~~~~~Section \ref{c3}
{[}3{]}. Contribution ~~~~~~Section \ref{c4} {[}4{]}. Simulation \&
Tests ~~~~~~Section \ref{c5} {[}5{]}. Conclusion \& Perspectives 

     {Section \ref{toc} - Section \ref{toc} - Page 01 -
Section \ref{archpar} ~~~~ - Phd : DAG Scheduling with Data Placement on
NUMA Multicore Platform}

DAG scheduling with Data placement on NUMA platform

\begin{longtable}[]{@{}l@{}}
\toprule
\bottomrule
\end{longtable}

\textbar{}

\emph{How to run DAG based parallel application on NUMA architecture
~~~~~~ while ~~~~~~ ~~~~~~ ~~~~~~ ~~~~~~ ~~~~~~ ~~~~~~ ~~~~~~ - we take
benefit from its scalabilty aspect ~~~~~~ ~~~~~~ ~~~~~~~~~~~~~~~~~ - we
avoid/reduce its penality as possible as we can ?~~~~~~ ~~~~~~}

\textbar{} {[}source :
https://www.supermicro.com/manuals/motherboard/5500/MNL-1083.pdf{]} 

     {Section \ref{toc} - Section \ref{pb0} - Page 02 - Section \ref{numa}
~~~~ - Phd : DAG Scheduling with Data Placement on NUMA Multicore
Platform}

Single, MultiCore CPUs, UMA/NUMA platforms

\begin{itemize}
\tightlist
\item
  Single Core CPU :Moore, Performance race and Three walls (Free Lunch).
\item
  Multicore revolution, (Manycore next) : Free lunch is Over 
\item
  Memory organisation based architectures UMA / NUMA
\item
  Uniform Memory Access : Shared Bus and Scalability problem 
\end{itemize}

     {Section \ref{toc} - Section \ref{archpar} - Page 03 -
Section \ref{dag0} ~~~~ - Phd : DAG Scheduling with Data Placement on
NUMA Multicore Platform}

Non Uniform Memory Access NUMA

 

     {Section \ref{toc} - Section \ref{numa} - Page 04 - Section \ref{task}
~~~~ - Phd : DAG Scheduling with Data Placement on NUMA Multicore
Platform}

Parallel applications \& Task Graph

Functionnal Parallelism

\begin{itemize}
\tightlist
\item
  Task based parallel programming model (paradigm), 
\item
  Parallelization Processus based on FOSTER approach :
\end{itemize}

 

     {Section \ref{toc} - Section \ref{dag0} - Page 05 - Section \ref{dpd}
~~~~ - Phd : DAG Scheduling with Data Placement on NUMA Multicore
Platform}

Example

Let \(\Phi\) be the sequential algorithm of Linear equations system
solving to parallize : \[
x,b \in \mathbb{R}^n, A \in \mathbb{R}^{n \times n} \text{  (down side Triangular Matrix ) :} , A.x = B
\]

1- Tasks decompositon (Computation)

\begin{itemize}
\tightlist
\item
  What are the tasks in \(\Phi\) , its size (Granularity Coarse / Thin)
  : \$ \textbf{TK}\_\{rk\} : \text{ code_block } \$
\item
  Result of this step the tasks set \(T^*(\Phi)\)
\end{itemize}

Example Demo

 

     {Section \ref{toc} - Section \ref{task} - Page 06 - Section \ref{sch}
~~~~ - Phd : DAG Scheduling with Data Placement on NUMA Multicore
Platform}

2- Dependency analysis (Communication)

The task \(T\) is characterized by : \(In(T)\) (\(Out\)) : read (write)
variables set - Shared variables :
\(SV(T,T') = ((Out(T) \cap Out(T')) \cup (Out(T) \cap In(T')) \cup (Out(T') \cap In(T))\)
- Dependent Tasks : share variables
\(T \bot T' \iff SV(T,T') \neq \emptyset\) - Precedence Relation :
execution causality : \(T \prec T' \iff (T \bot T') \& T \leq_{seq} T'\)
- \(\prec\) defines Partial order on \(T^*(\Phi)\) - DAG : This order is
represented by directed acyclic graphe (DAG)
\(G = (\mathbb{T},\mathbb{E})\) such : \(\mathbb{T}\) = Tasks set
\(T^*(\Phi)\). and \(e=(T,T') \in \mathbb{E} \iff T \prec T'\)

Example Demo : \(T_{1,2} : b_2 = b_2 - a_{1,2}x_1\) \&
\(T_{1,1} : x_1 = \frac{b_1}{a_{1,1}}\) we have
\(T_{1,1} \prec T_{1,2}\)

 

     {Section \ref{toc} - Section \ref{dpd} - Page 07 -
Section \ref{penalitynuma} ~~~~ - Phd : DAG Scheduling with Data
Placement on NUMA Multicore Platform}

Let DAG \(G(V,E,w,c)\) task graph describing a given parallel
application to run on the platform \(\mathbb{P}\)

3- Scheduling

\(\textbf{Task Scheduling}\) of \(G\) is the function \(\theta\) that
maps each task to its \(\textbf{start time}\) :

\begin{align*}
  \theta \colon \mathbb{T} & \to \mathbb{R}\\
  T                    & \mapsto t_s = \theta(T)\\
  s.t & \quad \forall (T, T') \in \mathbb{E}, \quad \theta(T) + w(T) < \theta(T')
\end{align*}

4- Mapping

Execution Resource : Processor / Core / Node / Thread

\(\textbf{Processor Mapping}\) is function \(\pi\) that maps to each
task to a execution resource to run on \(\mathbb{P}\) :

\begin{align*}
  \pi \colon \mathbb{T} & \to \mathbb{P}\\
  v_i                    & \mapsto p_j = \pi(v_i)\\
  s.t & \quad \forall (T,T') \in \mathbb{T}^2, \pi(T) = \pi(T') \implies (\theta(T)+ w(T) < \theta(T')) \quad or \quad (\theta(T')+ w(T') < \theta(T))
\end{align*}

 

     {Section \ref{toc} - Section \ref{sch} - Page 08 - Section \ref{state}
~~~~ - Phd : DAG Scheduling with Data Placement on NUMA Multicore
Platform}

NUMA Penality and Data Locality / Affinity

 

     {Section \ref{toc} - Section \ref{penalitynuma} - Page 09 -
Section \ref{loadbal} ~~~~ - Phd : DAG Scheduling with Data Placement on
NUMA Multicore Platform} NUMA

State of Art DAG Tasks Scheduling \& Data Placement on NUMA

\textbar{}

Author

\begin{verbatim}
         | <h3>Scheduling/Mapping Policy</h3>                            |  Type of placement  | Src of PLC decision  | Scheduling entity  | ToD   | NCPS| Impl LY | DoRD    | SUDT | GRANU      |
\end{verbatim}

\textbar{}
-\/-\/-\/-\/-\/-\/-\/-\/-\/-\/-\/-\/-\/-\/-\/-\/-\/-\/-\/-\/-\/-\/-\/-\/-\/-
\textbar{}
-\/-\/-\/-\/-\/-\/-\/-\/-\/-\/-\/-\/-\/-\/-\/-\/-\/-\/-\/-\/-\/-\/-\/-\/-\/-\/-\/-\/-\/-\/-\/-\/-\/-\/-\/-\/-\/-\/-\/-\/-\/-\/-\/-\/-\/-\/-\/-\/-\/-\/-\/-\/-\/-\/-\/-\/-\/-\/-\/-\/-
\textbar{} -\/-\/-\/-\/-\/-\/-\/-\/-\/-\/-\/-\/-\/-\/-\/-\/-\/-\/-
\textbar{} -\/-\/-\/-\/-\/-\/-\/-\/-\/-\/-\/-\/-\/-\/-\/-\/-\/-\/-
\textbar{} -\/-\/-\/-\/-\/-\/-\/-\/-\/-\/-\/-\/-\/-\/-\/-\/- \textbar{}
-\/-\/-\/-\/-\/- \textbar{} -\/- \textbar{}-\/- \textbar{}-\/-
\textbar{}-\/- \textbar{}-\/- \textbar{} \textbar{}

Henrik/2014

\begin{verbatim}
   | <h3>AFFINITY ON TOUCH (OS)</h3>                               |  -                  |  -                  |  -                |  -     |X-X- | LIB/OS  | PROGR   | ANY  | PG         |
\end{verbatim}

\textbar{}

Dashti/2011

\begin{verbatim}
   | <h3><a href="docs/CARREFOUR01.pdf">CARREFOUR</a></h3>         |  THD clustring   |  OS/PMU            |  OS THD       |  EXE   |X-XX | OS      | PRFLG   | ANY  | PG         |
\end{verbatim}

\textbar{}

Pousa/2010

\begin{verbatim}
    | <h3><a href="docs/mai01.pdf">MAI</a></h3>                     |  THD PIN     |  -                 |  P-THD        |  SoE   |X-X- | LIB     | PROGR   | ARY  | BLK/PG     |
\end{verbatim}

\textbar{}

Ribeiro/2015

\begin{verbatim}
  | <h3><a href="docs/minas01.pdf">MINAS + profiling</a></h3>     |  Co-SCH      |  DTSH          |  P-THD        |  SoE   |XXX- |         | PPG/LIB | PPR  | ARY/BLK/PG |
\end{verbatim}

\textbar{}

Marathe/2011

\begin{verbatim}
  | <h3>FBoPP</h3>                                                |  THD PIN     |  -                 |  P-THD        |  SoE   |X-X- | LIB     | PRFLG   | ANY  | PGs        |
\end{verbatim}

\textbar{}

Nikolopoulos/2008

\textbar{}

SCHEDULE REUSE

\begin{verbatim}
 |  Loop SCH    |  DTDS  |  ITR   |  EXE   |X-XX | CPL+RT  | PROGR   | ARY  | ELT       |
\end{verbatim}

\textbar{}

Yoo/2012

\begin{verbatim}
      | <h3><a href="docs/yoo01.pdf">Unstructured parallelism</a></h3>|  TSK PLC        |  DTSH       |  TSK              |  SoE   |-X-X | RT      | -       | -    | -          |
\end{verbatim}

\textbar{}

Broquedis/2009

\begin{verbatim}
| <h3><a href="docs/FORESTGOMP01.pdf">FORESTGOMP</a></h3>       |  THD PLC   |  DTDS    |  OMP-THD  |  EXE   |XXXX | RT      | PROGR   | -    | -          |
\end{verbatim}

\textbar{}

Chen/2013

\begin{verbatim}
     | <h3><a href="">LAWS</a></a></h3>                              |  TSK PLC        |  DTDS   |  Cilk TSK         |  EXE   | XXXX | RT     | DAG     | ARY  | BLs        |
\end{verbatim}

Abbreviations : NCPS : NUMA/CACHE Optimisation; Scheduling / Placement -
SCH : Scheduling - ILY : Implementation Layer - DoRD : Decision Entity -
SUDT : supported Data Structure - THD Thread - TSK : Task - PLC :
Placement - LIB : Library - OS :Operating system - CPL : Compiler - RT :
Runtime support - PROGR : Programmer - PRFLG : Profeling - PIN : Pinning
- DTSH : Data Sharing - DTDS : Data Distribution - ITR : iteration

     {Section \ref{toc} - Section \ref{state} - Page 10 -
Section \ref{contributions} ~~~~ - Phd : DAG Scheduling with Data
Placement on NUMA Multicore Platform} NUMA

Load balancing strategies

\textbar{}

A - Work Sharing

\textbar{} \textbar{}

B - Work Stealing

\begin{longtable}[]{@{}lll@{}}
\toprule
\bottomrule
\end{longtable}

\textbar{}

Scheduler attemps to migrate threads to other processorshoping to
distribute work to underutilized processors

\textbar{} \textbar{}

Underutilized processors take the initiativeby stealing threads from
other processors

     {Section \ref{toc} - Section \ref{loadbal} - Page 11 -
Section \ref{indpdtk} ~~~~ - Phd : DAG Scheduling with Data Placement on
NUMA Multicore Platform}

Contributions

1- Scheduling / Placement generic schemas

     {Section \ref{toc} - Section \ref{contributions} - Page 12 -
Section \ref{exehorizon} ~~~~ - Phd : DAG Scheduling with Data Placement
on NUMA Multicore Platform}

Impact of policies combination on independant tasks

\(\mathbb{T} = \{ T_i \}_{i \in I}\) independant tasks set sharing same
data Current Task \(T_k\) to be scheduled with its data :

Scheduling Policy : - Round Robin - Less Loaded

Placement Policy : - Round Robin - Affinity on First touch

\textbar{}

Scheduling ~Placement

\textbar{}

Round Robin policy

\begin{verbatim}
       |  <h3>Less Load</h3>          |
\end{verbatim}

\textbar{}-\/-\/-\/-\/-\/-\/-\/-\/-\/-\/-\/-\/-\/-\/-\/-\/-\/-\/-\/-\/-\/-\/-\/-\/-\textbar{}-\/-\/-\/-\/-\/-\/-\/-\/-\/-\/-\/-\/-\/-\/-\/-\/-\/-\/-\/-\/-\/-\/-\/-\/-\/-\/-\/-\/-\/-\/-\/-\/-\/-\/-\/-\/-\/-\/-\/-\/-\/-\textbar{}-\/-\/-\/-\/-\/-\/-\/-\/-\/-\/-\/-\/-\/-\/-\/-\/-\/-\/-\/-\/-\textbar{}
\textbar{}

Round Robin

\textbar{}

\[N^*[k \quad mod \quad |N^*|]\]

\begin{verbatim}
       |   <h3>$$\text{argMin}_{a \in A^*} \text{getProcessorNbOfUses(a)}$$</h3>                 |
\end{verbatim}

\textbar{}

Affinity on First touch

\textbar{}

\[\text{argMin}_{a \in A^*} \text{getProcessorsLoad(a)}\]

\begin{verbatim}
      |                   <h3>$$M^*[ k \quad mod \quad |M^*|]$$</h3>  |
\end{verbatim}

     {Section \ref{toc} - Section \ref{indpdtk} - Page 13 -
Section \ref{dagpartitionning} ~~~~ - Phd : DAG Scheduling with Data
Placement on NUMA Multicore Platform}

NUMA Adapted Execution Horizon for DAG-Applications

DAG \(G(V^*,E^*)\) partionning

\[
\Phi(G) = (\{ G_i = (V^*_i, E^*_i) \}_{i:1 \cdots K}, Mx_{K \times K}) \\
G_i \text{ vérifiant } \bigcup_{1 \le i \le K} V^*_i = V^* \&  V^*_i \cap V^*_j = \emptyset \text{ , } \forall i \neq j
\] 

     {Section \ref{toc} - Section \ref{exehorizon} - Page 14 -
Section \ref{xxh} ~~~~ - Phd : DAG Scheduling with Data Placement on
NUMA Multicore Platform}

Set Preferences

\[
\Phi(\{ G_1, G_2, \cdots , G_n\}*, \{N_j\}) = \{(G_{i_1},N_{j_1}), (G_{i_1},N_{j_1}), \cdots, (G_{i_n},N_{j_n}) \}
\] Partitionned DAG Example 

     {Section \ref{toc} - Section \ref{dagpartitionning} - Page 15 -
Section \ref{xxh2} ~~~~ - Phd : DAG Scheduling with Data Placement on
NUMA Multicore Platform}

Apply Extended Execution Horizon

\(\mathbb{T}\) already partitionned into four catégories : \$\mathbb{T}
= T\_C \cup T\_X \cup T\_R \cup T\_Q \$

\textbar{}

Completed Tasks

\textbar{}

Running Tasks

\textbar{}

Ready Tasks

\textbar{}

Not ready yet Tasks

\begin{longtable}[]{@{}llll@{}}
\toprule
\bottomrule
\end{longtable}

\textbar{}

\(T_C\)

\textbar{}

\(T_X\)

\textbar{}

\(T_R = \{ T \in \mathbb{T} \quad | \quad \forall \text{Parent}(T) \in T_C \}\)

\textbar{}

\(T_Q = \{ T \in \mathbb{T} \quad | \quad \exists \text{Parent}(T) \notin T_C \}\)

\textbar{} \textbar{}

Idea : Not enough information about DAG execution!! Extend visibility
horizon, Partion more the unclassified tasks :
\(T_Q = T_V \cup T_H \cup T_F \cup T_U\)

\textbar{}

Completed Tasks

\textbar{}

Running Tasks

\begin{longtable}[]{@{}ll@{}}
\toprule
\bottomrule
\end{longtable}

\textbar{}

\begin{itemize}
\tightlist
\item
  \(T_V = \{ T \in \mathbb{T} | \forall \text{Parent}(T) \in T_C \cup T_X \text{ et } \exists \text{Parent}(T) \in T_X \}\)-
  \(T_H = \{ T \in \mathbb{T} | \forall \text{Parent}(T) \in T_C \cup T_X \cup T_R \text{ et } \exists \text{Parent}(T) \in T_R \}\)-
  \(T_F = \{ T \in \mathbb{T} | \exists \text{Parent}(T) \notin T_C \cup T_X \cup T_V \text{ et } \exists \text{Parent}(T) \in T_V \}\}\)-
  \(T_U = \{ T \in \mathbb{T} | \exists \text{Parent}(T) \notin T_C \cup T_X \cup T_R \cup T_V \}\)

  \textbar{}\textbar{} 
\end{itemize}

     {Section \ref{toc} - Section \ref{xxh} - Page 16 - Section \ref{xxh3}
~~~~ - Phd : DAG Scheduling with Data Placement on NUMA Multicore
Platform}

Apply Extended Execution Horizon

\(\textbf{Algorithm XH-VHFU}\)

 

     {Section \ref{toc} - Section \ref{xxh2} - Page 17 - Section \ref{xxh4}
~~~~ - Phd : DAG Scheduling with Data Placement on NUMA Multicore
Platform}

Apply Extended Execution Horizon

At runtime, Scheduler must take a decision \(D_c\) after the realese of
an execution ressource :

\[D_c = \Phi(P_f , A_f, H_z)\]

\begin{itemize}
\tightlist
\item
  \(\textbf{Preference}\) \(P_f\) : node bind to subDAG of the current
  task.
\item
  \(\textbf{Affinity}\) \(A_f\) : node where the most amount the task
  data loacated.
\item
  \(\textbf{Horizon}\) \(H_z\) : How much wide become the horizon after
  choisng task. \[
  \Delta H^*(T_r) = \alpha \Delta V(T_r) + \beta \Delta H(T_r) + \gamma \Delta F(T_r)  + \theta \Delta U(T_r) 
  \]
\end{itemize}

1- \(\textit{Ready task selection }\) : \(\textbf{optimize the XH : }\)
\(\Delta H^*(T_r)\).

\[
T_s =  \text{argMax}_{T_r \in T_R} \Delta H^*(T_r) 
\]

 2- \(\textit{Node allocated}\) :
\(\textbf{preserving the data locality}\) or
\(\textbf{already run the parent task}\).

\[
N_s^(T) =
\begin{cases}
SDN(T), & \text{if } getLDx(SDN(T),T) > DL\_THRESHOLD\\
\text{argMax}_{N \in N^*} getLDx(N, T), & \text{else}\\
\end{cases}
\] Functions and Parameters : - \(\textbf{getLocalityDistance}\) :
compute for \(T\) its locality distance if it is already allocated to
node \(N\). - \(\textbf{SDN}\) : subDAGNode to get the node to which
\(T_s\) subDAG bound (preference). - \(\textbf{DL_THRESHOLD}\) :
Threshold to decide if \(T_s\) stay on preference node or search another
allocation - \(\textbf{getLDx}\) : quantify the data localité to \(T_s\)
for selected node (distance Matrix \& current Placement).

 3- \(\textit{Data locality}\) : \(\textbf{Affinity_on_first_Touch}\) no
migration no modification. Memory allocation for request done by \(T_s\)
during its exécution will requested from its node memory.

     {Section \ref{toc} - Section \ref{xxh3} - Page 18 - Section \ref{dbws}
~~~~ - Phd : DAG Scheduling with Data Placement on NUMA Multicore
Platform}

Apply Extended Execution Horizon

At runtime, Scheduler must take a decision \(D_c\) after the realese of
an execution ressource :

     {Section \ref{toc} - Section \ref{xxh4} - Page 19 - Section \ref{sim}
~~~~ - Phd : DAG Scheduling with Data Placement on NUMA Multicore
Platform}

Distance based Work stealing for LB on NUMA

 

     {Section \ref{toc} - Section \ref{dbws} - Page 20 -
Section \ref{simcombin} ~~~~ - Phd : DAG Scheduling with Data Placement
on NUMA Multicore Platform}

Simulation \& Results analysis

1st Contribution : NUMA simulator HLSMN

Simulator Architecture : - Simulation Kernel - Topology Generator -
Application Generator/Loader

 

     {Section \ref{toc} - Section \ref{sim} - Page 21 - Section \ref{simxxh}
~~~~ - Phd : DAG Scheduling with Data Placement on NUMA Multicore
Platform}

2d Contribution : Policies combinition for inpendant Tasks scenarios

Settings

\textbar{}

Tology

\textbar{}

Application

\textbar{}

Scheduling

\textbar{}

Placement

\textbar{}

Metrics

\begin{longtable}[]{@{}lllll@{}}
\toprule
\bottomrule
\end{longtable}

\textbar{}

2 Node/Cores:2/2 DistanceIntra/Inter:10/100

\textbar{}

\(\textbf{T50}\) \& \(\textbf{T250}\) (N=50/250 indpt tasks sharing
data)

\textbar{}

\(\textit{Round Robin (SRR)}\) \(\textit{Load Balanced (SLL)}\)

\textbar{}

\(\textit{First Touch (PFT)}\) \(\textit{Round Robin(PRR)}\)

\textbar{}

\(C_{max}\) Completion time Remote/Local acceses ratio

\textbar{}

Results

Comments

\textbar{}

Results \textbar{}

PPR/SRR

\textbar{}

PRR/SLL

\textbar{}

PFT/SRR

\textbar{}

PFT/SLL

\textbar{}

Commengts

\begin{longtable}[]{@{}llllll@{}}
\toprule
\bottomrule
\end{longtable}

\textbar{}

50 TK

\textbar{}

7.854

\textbar{}

11.372

\textbar{}

6.018

\textbar{}

7.213

\textbar{}

SLL best scheduling

\textbar{} \textbar{}

250TK

\textbar{}

25.799

\textbar{}

40.107

\textbar{}

21.877

\textbar{}

39.084

\textbar{}

PFT best placement

\textbar{}

     {Section \ref{toc} - Section \ref{simcombin} - Page 22 -
Section \ref{simxxhcmax} ~~~~ - Phd : DAG Scheduling with Data Placement
on NUMA Multicore Platform}

3rd Contribution : XH-XVHFU heuristic scheduling/placement DAG

Settings

\textbar{}

Tology

\textbar{}

DAG

\textbar{}

Scheduling

\textbar{}

Placement

\textbar{}

Metrics

\begin{longtable}[]{@{}lllll@{}}
\toprule
\bottomrule
\end{longtable}

\textbar{}

8N2C, 4N4C, 8N1C4N2C, 2N4C, 2N2C

\textbar{}

DAG0Xx format STGwith Xx tasksDAG25, DAG50, DAG75DAG100, DAG125, DAG150

\textbar{}

\(\textit{With XH-XVHFU}\) \(\textit{Without XH-XVHFU}\)

\textbar{}

\(\textit{First Touch (PFT)}\)

\textbar{}

1-\(C_{max}\) Completion time2- NUMA Penality Ratio

\textbar{}

     {Section \ref{toc} - Section \ref{simxxh} - Page 23 -
Section \ref{simxxhpnuma} ~~~~ - Phd : DAG Scheduling with Data
Placement on NUMA Multicore Platform}

Results for Completion execution time \(C_{max}\)

\textbar{}

DAG

\textbar{}

25

\textbar{}

50

\textbar{}

75

\textbar{}

100

\textbar{}

125

\textbar{}

150

\begin{longtable}[]{@{}lllllll@{}}
\toprule
\bottomrule
\end{longtable}

\textbar{}

Cases

\textbar{}

4/6

\textbar{}

5/6

\textbar{}

5/6

\textbar{}

5/6

\textbar{}

5/6

\textbar{}

4/6

\textbar{} \textbar{}

Deviation

\textbar{}

+4\%

\textbar{}

+6\%

\textbar{}

+7\%

\textbar{}

-3\%

\textbar{}

+4\%

\textbar{}

+3

\textbar{} \textbar{}

Range

\textbar{}

55 - 88

\textbar{}

75-112

\textbar{}

110 - 180

\textbar{}

190 - 290

\textbar{}

215 - 355

\textbar{}

255 - 360

Comments :

The results show that :

\begin{longtable}[]{@{}l@{}}
\toprule
\bottomrule
\end{longtable}

\textbar{}

The integration of heuristic XH has given in most cases (on average 4
out of 6 cases) a positive difference by improving the total execution
time between 3\% and 6 \% on average compared to the default strategy.

\textbar{} \textbar{}

The widing horizon, as the scheduling process progresses, helps to guide
the decision-making process to select the current tasks to be executed
by exploiting the visibility information of the tasks not yet executed.

     {Section \ref{toc} - Section \ref{simxxhcmax} - Page 24 -
Section \ref{simdbws} ~~~~ - Phd : DAG Scheduling with Data Placement on
NUMA Multicore Platform}

Results for NUMA Penality Ratio

Comments

\textbar{}

Node/Core

\textbar{}

2/2

\textbar{}

2/4

\textbar{}

4/2

\textbar{}

8/2

\textbar{}

8/1

\textbar{}

4/4

\begin{longtable}[]{@{}lllllll@{}}
\toprule
\bottomrule
\end{longtable}

\textbar{}

Cases

\textbar{}

5/6

\textbar{}

4/6

\textbar{}

4/6

\textbar{}

3/6

\textbar{}

3/6

\textbar{}

3/6

\textbar{} \textbar{}

Deviation

\textbar{}

+4\%

\textbar{}

+2\%

\textbar{}

-2\%

\textbar{}

-3\%

\textbar{}

+2\%

\textbar{}

+1

\textbar{} \textbar{}

Range\%

\textbar{}

70 - 78

\textbar{}

80 - 89

\textbar{}

80 - 89

\textbar{}

82 - 92

\textbar{}

82 - 89

\textbar{}

82 - 91

Results show that : The impact of the XH heuristic integration was not
significant to reduce the number of remote access in the tests with low
deviation compared to the default strategy.

     {Section \ref{toc} - Section \ref{simxxhpnuma} - Page 25 -
Section \ref{conclusion} ~~~~ - Phd : DAG Scheduling with Data Placement
on NUMA Multicore Platform}

4th Contribution : Distance based Work stealing heuristic

Settings

\textbar{}

Tology

\textbar{}

Application

\textbar{}

Load Balancing

\textbar{}

Metrics

\begin{longtable}[]{@{}lllll@{}}
\toprule
\bottomrule
\end{longtable}

\textbar{}

Node/Cores:4/1

\textbar{}

\(\textbf{T1500}\) indpt tasks sharing data

\textbar{}

With/without DbWS heuristic

\textbar{}

Average Load

\textbar{}

Results

Results

\textbar{}

Case

\textbar{}

Option

\textbar{}

Max Average Load

\textbar{}

Min Average Load

\begin{longtable}[]{@{}llll@{}}
\toprule
\bottomrule
\end{longtable}

\textbar{}

A \textbar{}

Without WS

\textbar{}

70

\textbar{}

50

\textbar{} \textbar{}

B \textbar{}

With WS

\textbar{}

70

\textbar{}

50

\textbar{} \textbar{}

C \textbar{}

With DbWS

\textbar{}

52

\textbar{}

28

Comments

\begin{itemize}
\tightlist
\item
  The first two strategies (A and B) are almost similar and their
  evolution is increasing (maximum peak of 50).
\item
  The DbWS strategy (C) gives an average load more stable and much lower
  than the previous one (max peak of 35).
\item
  An average improvement of the load = 15 \% (maximum peak).
\end{itemize}

     {Section \ref{toc} - Section \ref{simdbws} - Page 26 -
Section \ref{productions} ~~~~ - Phd : DAG Scheduling with Data
Placement on NUMA Multicore Platform}

Conclusion \& Perspectives

{[}5.1{]} Thesis - Studied problem : Scheduling of DAG based parallel
applicattion with data placement on NUMA platform. - Subproblem 1 :
Tasks scheduling \& mapping on NUMA nodes (processors/cores) -
Subproblem 2 : Data placement on nodes memories - Subproblem 3 : Load
balancing in this ddynamic contexte - Objectif : find near optimal
scheduling/placement in the contexte of NUMA for DAG with balanced load

{[}5.2{]} Results - independant tasks : Scheduling of DAG based parallel
applicattion with data placement on NUMA platform. - NUMA Extended
Execution Horizon : Tasks scheduling \& Data placement on NUMA nodes
(processors/cores) 1- Total execution time (Cmax) : improvement of 6\%
compared to the default strategy. 2- NUMA penality : no significant
improvement less then 2\% - Distance based Work Stealing : Improvement
of 15\% compared to without WS or with classical WS

{[}5.3{]} Limits - Simulation : Scheduling of DAG based parallel
applicattion with data placement. - Statistics : Tasks scheduling \&
mapping on NUMA nodes (processors/cores) - Synthetic DAGs : Data
placement on nodes memories

{[}5.4{]} Perspectives - No Simulation, Real NUMA : Scheduling of DAG
based parallel applicattion with data. - Using Real Applications : Tasks
scheduling \& mapping on NUMA nodes (processors/cores) - Theoritical
Analysis : Data placement on nodes memories

     {Section \ref{toc} - Section \ref{conclusion} - Page 27 -
Section \ref{thanks} ~~~~ - Phd : DAG Scheduling with Data Placement on
NUMA Multicore Platform}

Scientific Productions

\textbar{}

Num

\textbar{}

Year

\textbar{}

Authors

\textbar{}

Paper title

\textbar{}

\begin{verbatim}
   Organization  </h3>|
\end{verbatim}

\textbar{} -\/-\/-\textbar{} -\/-\/-\/-\/-\textbar{}
-\/-\/-\/-\/-\/-\/-\/- \textbar{} -\/-\/-\/-\/-\/-\/-\/-\/-\/-\/-\/-
\textbar{}-\/-\/-\/-\/-\/-\/-\/-\/-\/-\/-\/-\/-\/-\/-\/-\/-\/-\/-
\textbar{} \textbar{}

01

\textbar{}

2013

\textbar{}

M.Slimane, A.Benmahdjoub, S.Benhadja

\textbar{}

DAG Automatic parallel code generation using MS TPL library on multicore
machines.

\textbar{}

INFODAYs TIC National Conference, CHLEF University

\textbar{} \textbar{}

02

\textbar{}

2014

\textbar{}

M.Slimane, L. Sekhri

\textbar{}

Pipelined Parallel Implementation of Cryptographic DES algorithm on
Multicore platform

\textbar{}

PhD-Days (Journee doctorale) LAPECI laboratoiry Oran

\textbar{} \textbar{}

03

\textbar{}

2015

\textbar{}

M.Slimane, L. Sekhri

\textbar{}

Modeling the Scheduling Problem of Identical Parallel Machines with Load
Balancing by Time Petri Nets

\textbar{}

Journal of Intelligent Systems and Applications

\textbar{} \textbar{}

04

\textbar{}

2015

\textbar{}

M.Slimane, L. Sekhri, Y. Cherifé

\textbar{}

HLSMN High Level Simulator of Multicore NUMA Achitecture

\textbar{}

\begin{verbatim}
   1st national conference on Embedded and Distributed Systems EDiS Oran<br></h3>|
\end{verbatim}

\textbar{}

05

\textbar{}

2017

\textbar{}

M.Slimane, L. Sekhri

\textbar{}

HLSMN High Level NUMA Simulator

\textbar{}

EEA Journal 3-2017 (www.eea-journal.ro)

     {Section \ref{toc} - Section \ref{productions} - Page 28 ~~~~ - Phd :
DAG Scheduling with Data Placement on NUMA Multicore Platform}

 


    % Add a bibliography block to the postdoc
    
    
    
    \end{document}
